\documentclass[zihao = -4,cn]{oucart}
\usepackage{hyperref}   % 设置目录和引用超链接, 取消backref


% 设置页边距
% \usepackage{geometry}
% \geometry{left=2.5cm,right=2.0cm,top=2.5cm,bottom=2.5cm}

\title{题目}
\entitle{English Title}
\author{姓名}
\studentid{学号}
\advisor{指导}
\department{学院}{2016级}
\dateyear{2020}
\datemonth{04}
\dateday{01}


\cnabstractkeywords{
 这是中文摘要这是中文摘这是中文摘这是中文摘这是中文摘。 这是中文摘要这是中文摘这是中文摘这是中文摘这是中文摘。 这是中文摘要这是中文摘这是中文摘这是中文摘这是中文摘。 这是中文摘要这是中文摘这是中文摘这是中文摘这是中文摘。 这是中文摘要这是中文摘这是中文摘这是中文摘这是中文摘。 这是中文摘要这是中文摘这是中文摘这是中文摘这是中文摘。 这是中文摘要这是中文摘这是中文摘这是中文摘这是中文摘。 这是中文摘要这是中文摘这是中文摘这是中文摘这是中文摘。 这是中文摘要这是中文摘这是中文摘这是中文摘这是中文摘。 这是中文摘要这是中文摘这是中文摘这是中文摘这是中文摘。 这是中文摘要这是中文摘这是中文摘这是中文摘这是中文摘。 这是中文摘要这是中文摘这是中文摘这是中文摘这是中文摘。 这是中文摘要这是中文摘这是中文摘这是中文摘这是中文摘。 这是中文摘要这是中文摘这是中文摘这是中文摘这是中文摘。 
}{
中文, 关键字
}
\enabstractkeywords{
  这是英文摘要正文, Accurate Image Annotation Algorithm based on Human-machine collaboration.
  Accurate Image Annotation Algorithm based on Human-machine collaboration.Accurate Image Annotation Algorithm based on Human-machine collaboration.Accurate Image Annotation Algorithm based on Human-machine collaboration.Accurate Image Annotation Algorithm based on Human-machine collaboration.Accurate Image Annotation Algorithm based on Human-machine collaboration.Accurate Image Annotation Algorithm based on Human-machine collaboration.Accurate Image Annotation Algorithm based on Human-machine collaboration.Accurate Image Annotation Algorithm based on Human-machine collaboration.Accurate Image Annotation Algorithm based on Human-machine collaboration.
}{
  English, Abstract
}

\begin{document}

\makecover   % 制作封面

\makesignature  % 制作签字页

\makeabstract  % 制作摘要

\thispagestyle{empty}  % 设置目录样式
\tableofcontents

\newpage   % 这是手动分页
\pagenumbering{arabic}
\setcounter{page}{1}   % 从当前开始页码计数
% 页码设置为宋体小5的话,要设置页脚了吧,用pagestyle那个,但是要怎么从当前页开始设置呢?

% 正文内容

\section{绪论}
\subsection{前言}
\subsection{研究进展}
测试文字。测试文字。测试文字。测试文字。测试文字。测试文字。测试文字。测试文字。测试文字。测试文字。测试文字。测试文字。测试文字。测试文字。测试文字。测试文字。测试文字。测试文字。测试文字。测试文字。测试文字。测试文字。测试文字。测试文字。测试文字。测试文字。测试文字。测试文字。测试文字。测试文字。测试文字。测试文字。测试文字。测试文字。测试文字。测试文字。测试文字。测试文字。测试引用\cite{ouc}. 
\section{表格图片和公式}
测试文字。
\begin{figure}[!htbp]
    \centering
    \includegraphics[width = 0.2\textwidth]{assets/logo}
    \caption{中国海洋大学}
    \label{fig:ouc1}
\end{figure}

Table:
\begin{table}[!htbp]
\centering
\caption{三线表}
\begin{minipage}[t]{350pt}
\begin{tabular*}{350pt}{@{\extracolsep{\fill}}ccc}
\toprule
第一列 & 第二列 & 第三列 \\
\midrule
文字 & English & $\alpha^*$ \\
文字 & English & $\beta$ \\
文字 & English & $\gamma$\\
\bottomrule
\end{tabular*}
\footnotesize
% 数据来源:相关的数据来源。 \\
% $*$:表中需要解释的内容
\end{minipage}
\end{table}

公式\cite{DBLP:journals/corr/DaiHLRS16}:
\begin{equation}
 \lim_{x\to 0}{\frac{e^x-1}{2x}}
 \overset{\left[\frac{0}{0}\right]}{\underset{\mathrm{H}}{=}}
 \lim_{x\to 0}{\frac{e^x}{2}}={\frac{1}{2}}
\end{equation}
\newpage



\section{模型方法概述}
\subsection{模型A}
\subsection{模型B}

\newpage

\section{实验设置}
\subsection{实验环境}
\subsection{实验细节}

\newpage

\section{实验结果}

\newpage

\section{实验分析}
\newpage

\section{总结与展望}

\newpage
%\bibliographystyle{unsrt}
\bibliography{main}

\newpage


\begin{center}
\zihao{3} \textbf{致谢} \\
\end{center}

\newpage
\begin{center}
\zihao{3} \textbf{附录} \\
\end{center}
\end{document}
